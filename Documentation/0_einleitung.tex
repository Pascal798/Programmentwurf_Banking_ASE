\chapter{Einleitung}
In diesem Programmentwurf wurde Finanz/Banking-Anwendung entwickelt. Das projekt besteht aus einem backend und einem Frontend. Für das Backend wurde eine .NET Core WebApi 
erstellt und für das Frontend ein einfacher Client mit Hilfe von Windows Forms. Der Client dient dazu die Funktionen der API auszuführen und darzustellen. Die WebApi 
greift auf Anfrage des Clients auf eine InMemory-Datenbank zu und führt auf dieser Operationen aus. Da es sich hier um eine InMemory-Datenbank handelt, die im Speicher der Api läuft, 
wird die Datenbank resettet, wenn die WebApi gestoppt wird. 
\newline Das Frontend und das Backend wurden mit der Visual Studio 2019 Community Edition erstellt.
\newline Um das Testen der WebApi durch den Client zu vereinfachen, wird bei Start des Clients automatisch ein Test-Benutzer angelegt. Die Anmeldedaten für diesen Testnutzer sind:
\begin{itemize}
    \item Email: Test@test.com
    \item Passwort: Passw0rd
\end{itemize}
Dieser Benutzer wird als Admin angelegt und kann somit alle Funktionen der Anwendung nutzen. Welche Funktionen nicht genutzt werden können, wird später erklärt.
\newline Weiterhin werden bei Start 3 Banken auf der Datenbank angelegt, um Konten anlegen zu können, ohne dass eine Bank ausgewählt werden muss.
\newline Zum eigentlichen Programmentwurf gehört nur die WebApi. Alle Vorgaben für den Programmentwurf wurden in dieser Api umgesetzt und nicht im Client-Projekt. 
Der Client hilft nur beim Testen der Funktionen der Api.
\newline Als Admin-Benutzer können im Client neue Banken angelegt werden, die daraufhin jedem anderen benutzer sichtbar sind. Diese Funktion ist nicht für Nicht-Admin-Benutzer 
verfügbar, da nicht alle Benutzer die Möglichkeit haben sollten, neue Banken einzutragen, da diese daraufhin in den Clients aller Benutzer erscheinen. So wird verhindert, dass 
normale Benutzer zu viele Banken eintragen, auf denen Konten angelegt werden können oder Dopplungen entstehen.. 
\section{Installation}
\subsection{Anforderungen}
Um die WebApi zu starten wird folgende Software benötigt:
\begin{itemize}
    \item .NET SDK 5.0
    \item Visual Studio 2019 oder Visual Studio Code
\end{itemize}
\subsection{Starten der Anwendung mit Visual Studio 2019}
Um die WebApi in Visual Studio 2019 zu starten, wird die gesamte Projektmappe mit Visual Studio geöffnet. Im Solution-Explorer sind daraufhin folgende Projekte zu sehen:
\begin{itemize}
    \item Programmentwurf\_BankingApi
    \item Programmentwurf\_Banking\_Client
    \item Programmentwurf\_Mock\_Tests
    \item Programmentwurf\_xUnit\_Tests
\end{itemize}
In \glqq Programmentwurf\_BankingApi \grqq befinden sich die einzelnen Projekte der WebApi. Durch Rechtsklick auf die Solution im Solution-Explorer kann unter \glqq Startprojekte festlegen \grqq 
ausgewählt werden, welche Projekte durch drücken von F5 gestartet werden sollen. Hier wird für das Projekt \glqq 0\_Plugin \grqq die Option \glqq Starten \grqq ausgewählt. Optional kann auch 
für das Projekt \glqq Programmentwurf\_Banking\_Client \grqq die Option \glqq Starten \grqq ausgewählt werden, um auch den Client direkt zu starten.
\subsection{Starten der Anwendung mit Visual Studio Code}
Um die WebApi in Visual Studio Code zu starten, muss das Projekt \glqq 0\_Plugin \grqq ausgeführt werden. Dafür wird in den Ordner \glqq Programmentwurf\_BankingApi \grqq navigiert und von dort in 
den Projektordner \glqq 0\_Plugin \grqq . Dort wird über die Terminal-Konsole folgende Befehle ausgeführt:
\begin{itemize}
    \item dotnet build
    \item dotnet run
\end{itemize}
Mit \glqq dotnet \grqq build wird versucht die Anwendung zu kompilieren. Dadurch werden auch alle NuGet-Pakete heruntergeladen, die für die Entwicklung des Porgramms genutzt wurden.
\newline Mit \glqq dotnet run \grqq wird daraufhin die Anwendung gestartet. Wird die Anwendung über Visual Studio Code gestartet, ist sie über den localhost mit Port 5001 erreichbar.
\subsection{Benutzen der Anwendung}
Die WebApi kann entweder durch Postman oder den dazugehörigen Client getestet werden. Im GitHub-Repository befindet sich eine postman-Collection die genutzt werden kann um die WebApi mit Postman zu testen.
\newline Bei Ausführen des Clients wird ein Login-Screen geöffnet. Von diesem aus kann entweder ein neuer Benutzer registriert werden oder es wird der Admin-Benutzer zum einloggen genutzt, dessen Anmeldedaten zuvor schon erwähnt wurden. 
Daraufhin öffnet sich der Home-Screen von dem aus alle Funktionen erreichbar sind. 
\newline Es muss beachtet werden, dass jedesmal wenn etwas erstellt oder hinzugefügt wird über den Client, der Aktualisierungs-Button geklickt werden muss, um die Daten im Client zu aktualisieren.